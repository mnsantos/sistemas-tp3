Durante el transcurso de este trabajo pr\'actico comprendimos los usos y ventajas del paradigma MapReduce para an\'alisis de datos en sistemas distribuidos, 
as\'i como tambien las ventajas de montar un proyecto sobre HaaS, evaluando distintas alternativas de dicho servicio.


Entendemos que, para la primera parte, MapReduce ofrece una gran herramienta al momento de efectuar ciertos an\'alisis sobre 
bases de datos muy extensas, dado que se puede paralelizar de manera distribuida sin perder precisi\'on en los resultados. A su vez, 
encadenando secuencialmente Map Reduces sobre las sucesivas colecciones que se van generando se pueden realizar profundos an\'alisis 
muy f\'acilmente y con peso computacional distribuido.


Aun as\'i, errores en el c\'odigo pueden llevar a resultados inesperadamente engañosos, 
por lo que es importante que las funciones $map$ y $reduce$ cumplan con ciertos requisitos (http://docs.mongodb.org/manual/reference/command/mapReduce/\#mapreduce-map-cmd), 
fundamentalmente idempotencia en la funci\'on $reduce$, que el orden de los valores no altere el resultado final, y que el valor de retorno 
de la funci\'on $reduce$ matchee con el valor emitido por $map$ (puesto que puede que se llame a $reduce$ mas de una vez para la misma key).

El modelo de procesamiento map reduced permite escalar la base de datos horizontalmente.

Al utilizar este paradigma, mongoDB puede explotar todos los recursos que brinda un cluster de procesamiento.

Para montar inicialmente la base de datos de \emph{Reddit} proponemos un cluster minimal que soporta sharding (2 query routers, 3 config servers y 2 shards).
La idea es ir agregando o quitando nodos a medida que la demanada aumenta o disminuye.

Alcanzada esta etapa abrimos el siguiente interrogante: ¿Conviene montar el cluster con hardware propio o utilizar web-services provistos por empresas como 
Amazon o google? El primer camino es riesgoso a nivel económico, ya que requiere una fuerte inversión inicial, poca flexibilidad para escalar 
junto con las fluctuaciones de la demanda 
y la necesidad de invertir
recursos en el mantenimiento de hardware. La segunda, por el contrario, es mucho más viable. La empresa puede obtener poder de procesamiento y almacenamiento
a través de máquinas virtuales por un bajo costo. Al tratarse justamente de componentes no físicos, el usuario puede pedir o liberar los recursos de forma
rápida y flexible. De esta manera termina pagando exactamente por los recursos que está utilizando, y no más que eso. En suma, se libera de la responsabilidad
de realizar cualquier tipo de tarea de mantenimiento. Los precios bajos y el buen rendimiento de este tipo de servicios permite que las aplicaciones (junto con
sus bases de datos) escalen tanto en performance como económicamente a medida que el volumen de los datos y de los clientes aumenta. Finalmente optamos por los 
servicios Ec2 de AWS debido a que ofrecen un soporte oficial para mongoDB.

Muchas aplicaciones y páginas webs importantes en la actualidad se han beneficiado con el surgimiento y desarrollo de ''cloud computing'' y del fenómeno 
''everything as a service'' (donde \emph{Haas} está incluído). Reddit es un ejemplo real. Una página web que soporta 4 billones de visitas diarias es capaz
de funcionar con menos de 20 empleados debido a que delega gran parte de las responsabilidades necesarias para el funcionamiento de su aplicación en los 
servicios de Amazon.


